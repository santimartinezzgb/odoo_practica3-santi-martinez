\documentclass[12pt]{article}
\usepackage[utf8]{inputenc}
\usepackage{geometry}
\geometry{margin=2cm}
\usepackage{graphicx}
\usepackage{hyperref}
\usepackage{media9}
\usepackage{listings}
\usepackage{xcolor}

\lstset{
    basicstyle=\ttfamily,
    keywordstyle=\color{blue}\bfseries,
    commentstyle=\color{green!50!black},
    stringstyle=\color{red},
    breaklines=true
}


\title{Documentación del módulo Hola Mundo en Odoo}
\author{Santi Martínez}
\date{\today}

\begin{document}
    \maketitle

    \newpage
    \renewcommand{\contentsname}{ÍNDICE}
    \tableofcontents

    % ========================== INTRODUCCIÓN ==========================
    \newpage
    \section{Introducción}

    % ========================== PREPARACIÓN DEL ENTORNO ==========================
    \newpage
    \section{Preparación del entorno}

    % ========================== ACTIVAR EL MODO DESARROLLADOR ==========================
    \newpage
    \section{Activar del modo desarrollador en Odoo}
        Antes de proceder a su activación, cabe saber que el modo desarrollador permite ver y modificar la estructura interna de Odoo; permisos indispensables para crear o depurar módulos.
        El modo desarrollador en Odoo se puede activar de tres maneras diferentes:
        \begin{enumerate}
            \item Desde la interfaz propia de Odoo.
            \begin{itemize}
                \item Entrar en Odoo con el usuario administrador.
                \item \textbf{Ajustes} $\rightarrow$ \textbf{Activar modo desarrollador} (El cual solo será visible si hay al menos un módulo instalado)
            \end{itemize}

            \item Desde la URL:
            \begin{itemize}
                \item Añadir al final de la URL: \texttt{?debug=1}\\
                Por ejemplo: \texttt{http://localhost:8069/web?debug=1}
            \end{itemize}

            \item Con una extensión en el navegador:
            \begin{itemize}
                \item Firefox: \url{https://addons.mozilla.org/es/firefox/addon/odoo-debug/}
                \item Chrome \url{https://chrome.google.com/webstore/detail/odoo-debug/hmdmhilocobgohohpdpolmibjklfgkbi?hl=es_PR}
            \end{itemize}
        \end{enumerate}

        En esta documentación se tratará la opción propia de la interfaz de Odoo, la número uno.


        \subsection{Entrar en Odoo con el usuario administrador}
            \begin{figure}[!htb]
                \centering
                \includegraphics[width=0.8\textwidth]{images/administrador.png}
                \caption{Odoo con usuario administrador}
            \end{figure}

        \subsection{Activar modo desarrollador}
            \begin{figure}[!htb]
                \centering
                \includegraphics[width=0.5\textwidth]{images/ajustes.png}
                \caption{Ajustes}
            \end{figure}
            Clicar en \textbf{Activar modo de desarrollador}
            \begin{figure}[!htb]
                \centering
                \includegraphics[width=1\textwidth]{images/modo-desarrollador.png}
                \caption{Activar modo desarrollador}
            \end{figure}

    % ========================== PRIMER MÓDULO: "HOLA MUNDO" ==========================
    \newpage
    \section{Primer módulo: "Hola mundo"}
        El objetivo de este primer módulo es el comprobar que Odoo detacta correctamente los módulos que se crean de forma manual.
        La ubicación de los módulos personalizados se guarda dentro del directorio configurado para módulos, en este caso dentro de /módulos:
        \begin{figure}[!htb]
            \centering
            \includegraphics[width=0.6\textwidth]{images/hola_mundo.png}
            \caption{Ubicación del módulo holamundo}
        \end{figure}

        En el interior de esta carpeta, se crean dos ficheros:
        \begin{itemize}
            \item \textbf{\_init\_.py}. El cual estará vacío
            \item \textbf{\_manifest\_.py}. Contiene el siguiente código:
                \begin{figure}[!htb]
                    \centering
                    \includegraphics[width=0.65\textwidth]{images/code_manifest.png}
                \end{figure}
        \end{itemize}

        El directorio \textbf{/módulos} queda así:
        \begin{figure}[!htb]
            \centering
            \includegraphics[width=0.65\textwidth]{images/init_manifest copy.png}
            \caption{ficheros}
        \end{figure}
        
        
        Una vez creada la estructura, con el "Modo desarrollador" activado, se actualizará la liasta de aplicaciones.
        \begin{enumerate}
            \item Entrar en el menú de Odoo e ir a la opción de \textbf{Aplicaciones}:
                \begin{figure}[!htb]
                    \centering
                    \includegraphics[width=0.35\textwidth]{images/entrar-aplicaciones copy.png}
                \end{figure}
                
            \item Ir a \textbf{Actualizar lista de aplicaciones}:
                \begin{figure}[!htb]
                    \centering
                    \includegraphics[width=0.8\textwidth]{images/actualizar-lista.png}
                \end{figure}

            \item Confirmar que \textbf{sí} se quiere actualizar:
                \begin{figure}[!htb]
                    \centering
                    \includegraphics[width=0.55\textwidth]{images/confirmar-actualizar.png}
                \end{figure}

            \item Muestra el módulo \textbf{"Hola mundo"}:
                \begin{figure}[!htb]
                    \centering
                    \includegraphics[width=1\textwidth]{images/modulo-ejemplo.png}
                    \caption{Aparición del módulo Hola mundo}
                \end{figure}
                
        \end{enumerate}


    % ========================== CREACIÓN DE MÓDULOS EN ODOO ==========================
    \newpage
    \section{Creación de módulos en Odoo}
        Crear un módulo en Odoo y establecer una aplicación propia integrada dentro de un ecosistema más amplio. Considerar cada módulo como una mini aplicación que forma parte de una aplicación madre mayor: Odoo. Diseñar el módulo de manera que funcione con independencia del resto, definiendo su propia lógica, modelos de datos, vistas, controladores y funcionalidades específicas. Aprovechar la estructura modular de Odoo para garantizar que cada componente opere de forma autónoma, manteniendo la coherencia del sistema general.

        Desarrollar cada módulo como una aplicación dentro de otra aplicación, asegurando autonomía y flexibilidad en su diseño. Utilizar esta organización modular para ampliar Odoo de manera escalable y personalizada, ajustando las funciones a las necesidades particulares de cada entorno sin alterar la base del sistema. Mantener la integración fluida entre módulos para que cada uno actúe como una pieza esencial dentro de un conjunto perfectamente engranado, reforzando así la adaptabilidad y el rendimiento global de la plataforma.
        \subsection{Creación de módulos con Odoo Scaffold}

        Utilizar el comando \texttt{odoo scaffold} para generar la estructura base de un nuevo módulo dentro del entorno de desarrollo de Odoo. Este comando permite crear automáticamente todos los archivos y carpetas necesarias para iniciar un módulo funcional, siguiendo la arquitectura estándar de Odoo.
        Para ello, introducir el comando: \[\texttt{odoo scaffold "nombre del modulo" "ruta"}\]

        Antes de ese paso, hay que instalar Odoo 16 por terminal, en este caso con el comando:
        \[\textbf{sudo apt install odoo-16}\]

        \begin{figure}[!htb]
            \centering
            \includegraphics[width=\textwidth]{images/instalacion_odoo-16.png}
            \caption{Instalación de Odoo 16}
        \end{figure}

        En este caso se crearán 5 módulos nuevos, orientados a la educación, en concreto a módulos enfocados en la organización personal del profesorado:

        \begin{itemize}
            \item agenda\_profesor
            \item clases
            \item asistencia\_alumnos
            \item calendario
            \item registro\_notas
        \end{itemize}

        \newpage
        \underline{Para ello se utilizan los siguientes comandos:}
        
        \begin{figure}[!htb]
            \centering
            \includegraphics[width=1\textwidth]{images/crear-modulos.png}
            \caption{código de creación de módulos con scaffold}
        \end{figure}
        \begin{figure}[!htb]
            \centering
            \includegraphics[width=1\textwidth]{images/instalaciones-modulos.png}
            \caption{Contenido de cada módulo}
        \end{figure}


        \newpage
        \begin{figure}[!htb]
            \centering
            \includegraphics[width=0.9\textwidth]{images/agenda_profesor.png}
            \caption{Contenido de cada módulo}
        \end{figure}
        Breve explicación de cada fichero generado en el interior de cada uno de los módulos:
        \begin{itemize}
            \item \textbf{models/models.py}: define un ejemplo del modelo de datos y sus campos.
            \item \textbf{views/views.xml}: describe las vistas de nuestro módulo (formulario, árbol, menús, etc.).
            \item \textbf{demo/demo.xml}: incluye datos ``demo'' para el ejemplo propuesto de modelo.
            \item \textbf{controllers/controllers.py}: contiene un ejemplo de controlador de rutas, implementando algunas rutas.
            \item \textbf{views/templates.xml}: contiene dos ejemplos de vistas ``qweb'' usado por el controlador de rutas.
            \item \textbf{\_\_manifest\_\_.py}: es el manifiesto del módulo. Incluye información como el título, descripción, así como ficheros a cargar.
        \end{itemize}



    % ========================== CREACIÓN DE LISTA DE TAREAS ==========================
    \newpage
    \section{Módulo funcional: "Lista de tareas"}
        \begin{figure}[!htb]
            \centering
            \includegraphics[width=1\textwidth]{images/crear-lista-tareas.png}
            \caption{Creación de Lista de tareas}
        \end{figure}

        \begin{figure}[!htb]
            \centering
            \includegraphics[width=1\textwidth]{images/cambiar-nombre-lista.png}
            \caption{Cambiar el nombre a \textbf{Lista de tareas} sin \_ }
        \end{figure}

        \begin{figure}[!htb]
            \centering
            \includegraphics[width=1\textwidth]{images/actualizar-modulo.png}
            \caption{Módulo en Odoo}
        \end{figure}




    \newpage
    \section{Modificación del módulo "Lista de tareas"}


        Este módulo permite crear y gestionar tareas en Odoo de forma sencilla.
        Cada tarea tiene un título, descripción, un valor numérico, un cálculo automático y un estado de completada.

        \subsection{Modelo de Datos}
        \begin{figure}[!htb]
            \centering
            \includegraphics[width=1\textwidth]{images/modelo-lista-tareas.png}
            \caption{Módulo en Lista de tareas}
        \end{figure}


        \subsection{Explicación de Campos}

        \begin{itemize}
            \item \textbf{nombre}: Título de la tarea, obligatorio.
            \item \textbf{descripcion}: Descripción o detalles de la tarea.
            \item \textbf{valor}: Número que se asigna a la tarea.
            \item \textbf{valor2}: Calculado automáticamente como value/100.
            \item \textbf{completada}: Indica si la tarea está completada.
        \end{itemize}

        Para poder aplicar estos cambios al módulo Lista de tareas, es necesario reiniciar los contenedores y en los 3 puntos del módulo en Odoo y darle click en \textbf{Actualizar}. Esto hará que todos esos cambios en el módulo sean vigentes.
        \begin{figure}[!htb]
            \centering
            \includegraphics[width=0.8\textwidth]{images/actualizar-modulo.png}
            \caption{Módulo en Lista de tareas}
        \end{figure}

    \newpage
    \section{Conclusiones}




\end{document}